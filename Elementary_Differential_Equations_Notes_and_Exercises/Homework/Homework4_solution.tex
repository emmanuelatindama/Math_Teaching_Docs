\documentclass[a4paper,11pt,reqno]{amsart}

\usepackage[utf8]{inputenc}
\usepackage[foot]{amsaddr}
\usepackage{amsmath,amsfonts,amssymb,amsthm,mathrsfs,bm}
\usepackage[margin=0.95in]{geometry}
\usepackage{color}
\usepackage[dvipsnames]{xcolor}

\usepackage{etoolbox}

% Modifications to amsart ToC-related macros...
\makeatletter
\let\old@tocline\@tocline
\let\section@tocline\@tocline
% Insert a dotted ToC-line for \subsection and \subsubsection only
\newcommand{\subsection@dotsep}{4.5}
\newcommand{\subsubsection@dotsep}{4.5}
\patchcmd{\@tocline}
  {\hfil}
  {\nobreak
     \leaders\hbox{$\m@th
        \mkern \subsection@dotsep mu\hbox{.}\mkern \subsection@dotsep mu$}\hfill
     \nobreak}{}{}
\let\subsection@tocline\@tocline
\let\@tocline\old@tocline

\patchcmd{\@tocline}
  {\hfil}
  {\nobreak
     \leaders\hbox{$\m@th
        \mkern \subsubsection@dotsep mu\hbox{.}\mkern \subsubsection@dotsep mu$}\hfill
     \nobreak}{}{}
\let\subsubsection@tocline\@tocline
\let\@tocline\old@tocline

\let\old@l@subsection\l@subsection
\let\old@l@subsubsection\l@subsubsection

\def\@tocwriteb#1#2#3{%
  \begingroup
    \@xp\def\csname #2@tocline\endcsname##1##2##3##4##5##6{%
      \ifnum##1>\c@tocdepth
      \else \sbox\z@{##5\let\indentlabel\@tochangmeasure##6}\fi}%
    \csname l@#2\endcsname{#1{\csname#2name\endcsname}{\@secnumber}{}}%
  \endgroup
  \addcontentsline{toc}{#2}%
    {\protect#1{\csname#2name\endcsname}{\@secnumber}{#3}}}%

% Handle section-specific indentation and number width of ToC-related entries
\newlength{\@tocsectionindent}
\newlength{\@tocsubsectionindent}
\newlength{\@tocsubsubsectionindent}
\newlength{\@tocsectionnumwidth}
\newlength{\@tocsubsectionnumwidth}
\newlength{\@tocsubsubsectionnumwidth}
\newcommand{\settocsectionnumwidth}[1]{\setlength{\@tocsectionnumwidth}{#1}}
\newcommand{\settocsubsectionnumwidth}[1]{\setlength{\@tocsubsectionnumwidth}{#1}}
\newcommand{\settocsubsubsectionnumwidth}[1]{\setlength{\@tocsubsubsectionnumwidth}{#1}}
\newcommand{\settocsectionindent}[1]{\setlength{\@tocsectionindent}{#1}}
\newcommand{\settocsubsectionindent}[1]{\setlength{\@tocsubsectionindent}{#1}}
\newcommand{\settocsubsubsectionindent}[1]{\setlength{\@tocsubsubsectionindent}{#1}}

% Handle section-specific formatting and vertical skip of ToC-related entries
% \@tocline{<level>}{<vspace>}{<indent>}{<numberwidth>}{<extra>}{<text>}{<pagenum>}
\renewcommand{\l@section}{\section@tocline{1}{\@tocsectionvskip}{\@tocsectionindent}{}{\@tocsectionformat}}%
\renewcommand{\l@subsection}{\subsection@tocline{1}{\@tocsubsectionvskip}{\@tocsubsectionindent}{}{\@tocsubsectionformat}}%
\renewcommand{\l@subsubsection}{\subsubsection@tocline{1}{\@tocsubsubsectionvskip}{\@tocsubsubsectionindent}{}{\@tocsubsubsectionformat}}%
\newcommand{\@tocsectionformat}{}
\newcommand{\@tocsubsectionformat}{}
\newcommand{\@tocsubsubsectionformat}{}
\expandafter\def\csname toc@1format\endcsname{\@tocsectionformat}
\expandafter\def\csname toc@2format\endcsname{\@tocsubsectionformat}
\expandafter\def\csname toc@3format\endcsname{\@tocsubsubsectionformat}
\newcommand{\settocsectionformat}[1]{\renewcommand{\@tocsectionformat}{#1}}
\newcommand{\settocsubsectionformat}[1]{\renewcommand{\@tocsubsectionformat}{#1}}
\newcommand{\settocsubsubsectionformat}[1]{\renewcommand{\@tocsubsubsectionformat}{#1}}
\newlength{\@tocsectionvskip}
\newcommand{\settocsectionvskip}[1]{\setlength{\@tocsectionvskip}{#1}}
\newlength{\@tocsubsectionvskip}
\newcommand{\settocsubsectionvskip}[1]{\setlength{\@tocsubsectionvskip}{#1}}
\newlength{\@tocsubsubsectionvskip}
\newcommand{\settocsubsubsectionvskip}[1]{\setlength{\@tocsubsubsectionvskip}{#1}}

% Adjust section-specific ToC-related macros to have a fixed-width numbering framework
\patchcmd{\tocsection}{\indentlabel}{\makebox[\@tocsectionnumwidth][l]}{}{}
\patchcmd{\tocsubsection}{\indentlabel}{\makebox[\@tocsubsectionnumwidth][l]}{}{}
\patchcmd{\tocsubsubsection}{\indentlabel}{\makebox[\@tocsubsubsectionnumwidth][l]}{}{}

% Allow for section-specific page numbering format of ToC-related entries
\newcommand{\@sectypepnumformat}{}
\renewcommand{\contentsline}[1]{%
  \expandafter\let\expandafter\@sectypepnumformat\csname @toc#1pnumformat\endcsname%
  \csname l@#1\endcsname}
\newcommand{\@tocsectionpnumformat}{}
\newcommand{\@tocsubsectionpnumformat}{}
\newcommand{\@tocsubsubsectionpnumformat}{}
\newcommand{\setsectionpnumformat}[1]{\renewcommand{\@tocsectionpnumformat}{#1}}
\newcommand{\setsubsectionpnumformat}[1]{\renewcommand{\@tocsubsectionpnumformat}{#1}}
\newcommand{\setsubsubsectionpnumformat}[1]{\renewcommand{\@tocsubsubsectionpnumformat}{#1}}
\renewcommand{\@tocpagenum}[1]{%
  \hfill {\mdseries\@sectypepnumformat #1}}

% Small correction to Appendix, since it's still a \section which should be handled differently
\let\oldappendix\appendix
\renewcommand{\appendix}{%
  \leavevmode\oldappendix%
  \addtocontents{toc}{%
    \protect\settowidth{\protect\@tocsectionnumwidth}{\protect\@tocsectionformat\sectionname\space}%
    \protect\addtolength{\protect\@tocsectionnumwidth}{2em}}%
}
\makeatother

% #1 (default is as required)

% #2

% #3
\makeatletter
\settocsectionnumwidth{2em}
\settocsubsectionnumwidth{2.5em}
\settocsubsubsectionnumwidth{3em}
\settocsectionindent{1pc}%
\settocsubsectionindent{\dimexpr\@tocsectionindent+\@tocsectionnumwidth}%
\settocsubsubsectionindent{\dimexpr\@tocsubsectionindent+\@tocsubsectionnumwidth}%
\makeatother

% #4 & #5
\settocsectionvskip{10pt}
\settocsubsectionvskip{0pt}
\settocsubsubsectionvskip{0pt}

% #6 & #7
% See #3

% #8
\renewcommand{\contentsnamefont}{\bfseries\Large}

% #9
\settocsectionformat{\bfseries}
\settocsubsectionformat{\mdseries}
\settocsubsubsectionformat{\mdseries}
\setsectionpnumformat{\bfseries}
\setsubsectionpnumformat{\mdseries}
\setsubsubsectionpnumformat{\mdseries}

% #10
% Insert the following command inside your text where you want the ToC to have a page break
\newcommand{\tocpagebreak}{\leavevmode\addtocontents{toc}{\protect\clearpage}}

% #11
\let\oldtableofcontents\tableofcontents
\renewcommand{\tableofcontents}{%
  \vspace*{-\linespacing}% Default gap to top of CONTENTS is \linespacing.
  \oldtableofcontents}

\usepackage{mathtools,enumerate,mathrsfs,graphicx}
\usepackage{epstopdf}
\usepackage{hyperref}

\usepackage{latexsym}


\definecolor{CommentGreen}{rgb}{0.0,0.4,0.0}
\definecolor{Background}{rgb}{0.9,1.0,0.85}
\definecolor{lrow}{rgb}{0.914,0.918,0.922}
\definecolor{drow}{rgb}{0.725,0.745,0.769}

\usepackage{listings}
\usepackage{textcomp}
\lstloadlanguages{Matlab}%
\lstset{
    language=Matlab,
    upquote=true, frame=single,
    basicstyle=\small\ttfamily,
    backgroundcolor=\color{Background},
    keywordstyle=[1]\color{blue}\bfseries,
    keywordstyle=[2]\color{purple},
    keywordstyle=[3]\color{black}\bfseries,
    identifierstyle=,
    commentstyle=\usefont{T1}{pcr}{m}{sl}\color{CommentGreen}\small,
    stringstyle=\color{purple},
    showstringspaces=false, tabsize=5,
    morekeywords={properties,methods,classdef},
    morekeywords=[2]{handle},
    morecomment=[l][\color{blue}]{...},
    numbers=none, firstnumber=1,
    numberstyle=\tiny\color{blue},
    stepnumber=1, xleftmargin=10pt, xrightmargin=10pt
}

\numberwithin{equation}{section}
\synctex=1

\hypersetup{
    unicode=false, pdftoolbar=true, 
    pdfmenubar=true, pdffitwindow=false, pdfstartview={FitH}, 
    pdftitle={ELE2024 Coursework}, pdfauthor={A. Author},
    pdfsubject={ELE2024 coursework}, pdfcreator={A. Author},
    pdfproducer={ELE2024}, pdfnewwindow=true,
    colorlinks=true, linkcolor=red,
    citecolor=blue, filecolor=magenta, urlcolor=cyan
}


% CUSTOM COMMANDS
\renewcommand{\Re}{\mathbf{re}}
\renewcommand{\Im}{\mathbf{im}}
\newcommand{\R}{\mathbb{R}}
\newcommand{\N}{\mathbb{N}}
\newcommand{\C}{\mathbb{C}}
\newcommand{\lap}{\mathscr{L}}
\newcommand{\dd}{\mathrm{d}}
\newcommand{\smallmat}[1]{\left[ \begin{smallmatrix}#1 \end{smallmatrix} \right]}

%opening
\title[MATH 2860 (Elementary Differential Equations)]{Homework 4 for MATH 2860 (Elementary Differential Equations)}

\author[Emmanuel Atindama]{E. A. Atindama, PhD Mathematics}

\address[E. A. Atindama]{. Email addresses: \href{emmanuel.atindama@utoledo.edu}{emmanuel.atindama@utoledo.edu} 
% and 
% \href{mailto:a.student@qub.ac.uk}{a.student@qub.ac.uk}.
}
\thanks{Version 0.0.1. Last updated:~\today.}

\begin{document}
\maketitle

Due at 10:00 EST in class

\subsection*{Question Q1}
  Solve the Bernoulli equation
  \[y' = y(xy^3 - 1), \quad y(1) = 1.\]

  \begin{center}\setlength{\fboxsep}{10pt}\fcolorbox{yellow!20}{yellow!20}{\parbox{0.9\linewidth}{
    \textbf{Solution}
    Rearranging the given Bernoulli equation, we have
    \[y' + y = xy^4.\]
    This is the standard form of a Bernoulli equation \(y' + P(x)y = Q(x)y^n\), with \(n = 4\).

    Make the substitution 
    \[
    v = y^{1-n} = y^{-3} \Rightarrow v' = -3y^{-4}y' \,\Rightarrow\, y^{-4}y' = -\frac{1}{3}v'
    \]
    So that the differential equation becomes linear in \(v\) given by
    \[
    -\frac{1}{3}v' + v = x.
    \]

    We can solve using integration factor to obtain
    \[
    e^{-3x}v=xe^{-3x}-\frac{1}{3}e^{-3x}+C \quad \Rightarrow \quad
    v=x-\frac{1}{3}+Ce^{3x}.
    \]
    Substitute back \(v = y^{-3}\) to obtain \(y^{-3}=x-\frac{1}{3}+Ce^{3x}\).

    Substitute the initial condition \(y(1)=1\) into the solution to find \(C\).
    \[1^{-3}=1-\frac{1}{3}+Ce^{3(1)}\quad \Rightarrow \quad 1=\frac{2}{3}+Ce^{3}\quad \Rightarrow \quad C=\frac{1}{3e^{3}}\].

    The solution to the Bernoulli equation is:
    \[y^{-3}=x-\frac{1}{3}+\frac{1}{3e^{3}}e^{3x} \quad
    \text{or} \quad
   \mathbf{y=}\left(\mathbf{x-}\frac{\mathbf{1}}{\mathbf{3}}\mathbf{+}\frac{\mathbf{e}^{\mathbf{3x-3}}}{\mathbf{3}}\right)^{\mathbf{-1/3}}\]
    }}
    \end{center}



\subsection*{Question Q2}
  Determine whether the differential equation 
  \[-y\,dx + (x+ \sqrt{x + y})\,dy = 0\]
  is homogeneous. If it is, solve it. If not, state the reason why it is not.
  \begin{center}\setlength{\fboxsep}{10pt}\fcolorbox{yellow!20}{yellow!20}{\parbox{0.9\linewidth}{
    \textbf{Solution}
    A differential equation of the form \(M(x,y)\,dx + N(x,y)\,dy = 0\) is homogeneous if both \(M\) and \(N\) are homogeneous functions of the same degree. A function \(f(x,y)\) is homogeneous of degree \(n\) if for all \(t > 0\),
    \[f(tx, ty) = t^n f(x,y).\]
    
    In our case, we have
    \[M(x,y) = -y \quad \text{and}\quad N(x,y) = x + \sqrt{x + y}.\]
    
    \[M(tx, ty) = -ty = t^1(-y) = t^1 M(x,y), \quad \text{so} M(x,y) \text{ is homogeneous of degree 1.}.\]
    
  
    \[N(tx, ty) = tx + \sqrt{tx + ty} = tx + \sqrt{t(x + y)} = tx + t^{1/2}\sqrt{x + y},\]
    
    which is not a single term multiplied by a power of \(t\). Therefore, \(N\) is not homogeneous.
    
    Since \(M\) and \(N\) are not both homogeneous functions of the same degree, the differential equation is not homogeneous.
    }}
    \end{center}

\subsection*{Question Q3}
  Determine whether the differential equation 
  \[-y\,dx + (x+ \sqrt{xy})\,dy = 0\]
  is homogeneous. If it is, solve it. If not, state the reason why it is not.
  \begin{center}\setlength{\fboxsep}{10pt}\fcolorbox{yellow!20}{yellow!20}{\parbox{0.9\linewidth}{
    \textbf{Solution}
    A differential equation of the form \(M(x,y)\,dx + N(x,y)\,dy = 0\) is homogeneous if both \(M\) and \(N\) are homogeneous functions of the same degree. A function \(f(x,y)\) is homogeneous of degree \(n\) if for all \(t > 0\),
    \[f(tx, ty) = t^n f(x,y).\]
    
    In our case, we have
    \[M(x,y) = -y \quad \text{and}\quad N(x,y) = x + \sqrt{xy}.\]
    
    \[M(tx, ty) = -ty = t^1(-y) = t^1 M(x,y), \quad \text{so} M(x,y) \text{ is homogeneous of degree 1.}.\]
    
  
    \[N(tx, ty) = tx + \sqrt{tx\cdot ty} = tx + \sqrt{t^2(xy)} = tx + t\sqrt{x + y},\]
    
    Hence, \(N\) is homogeneous of degree 1. So the differential equation is homogeneous.

    \vspace{0.5cm}

    \begin{minipage}{0.45\linewidth}
      Make the substitution
    \[y=\textcolor{magenta}{vx}\Rightarrow dy=\textcolor{magenta}{v\,dx+x\,dv}.\]

    \begin{align*}
      -y\,dx+(x+\sqrt{xy})\,dy &=0 \\
      -(\textcolor{magenta}{vx})\,dx+(x+\sqrt{x(\textcolor{magenta}{vx})})(\textcolor{magenta}{v\,dx+x\,dv}) &=0 \\
      -\textcolor{magenta}{vx}\,dx+(x+x\sqrt{\textcolor{magenta}{v}})(\textcolor{magenta}{v\,dx+x\,dv}) &= 0 \\
      -\textcolor{magenta}{vx}\,dx+x(1+\sqrt{\textcolor{magenta}{v}})(\textcolor{magenta}{v\,dx+x\,dv}) &= 0 \\
      -\textcolor{magenta}{v}\,dx+(1+\sqrt{\textcolor{magenta}{v}})(\textcolor{magenta}{\textcolor{magenta}{v}\,dx+x\,d\textcolor{magenta}{v}}) &= 0 \\
      -\textcolor{magenta}{v}\,dx+\textcolor{magenta}{v}(1+\sqrt{\textcolor{magenta}{v}})\,dx+x(1+\sqrt{\textcolor{magenta}{v}})\,d\textcolor{magenta}{v} &= 0 \\
      (\textcolor{magenta}{v}(1+\sqrt{\textcolor{magenta}{v}})-\textcolor{magenta}{v})\,dx+x(1+\sqrt{\textcolor{magenta}{v}})\,d\textcolor{magenta}{v} &= 0 \\
      (\textcolor{magenta}{v}+\textcolor{magenta}{v}^{3/2}-\textcolor{magenta}{v})\,dx+x(1+\sqrt{\textcolor{magenta}{v}})\,d\textcolor{magenta}{v} &= 0 \\
      \textcolor{magenta}{v}^{3/2}\,dx+x(1+\sqrt{\textcolor{magenta}{v}})\,dv &= 0
    \end{align*}
    which is separable.
    \end{minipage}
    \hfill
    \vline
    \hfill
    \begin{minipage}{0.45\linewidth}
      Thus,
    \begin{align*}
      \frac{1}{x}\,dx &=-\frac{1+\sqrt{v}}{v^{3/2}}\,dv\\
      \int \frac{1}{x}\,dx &=- \int \left(\frac{1}{v^{3/2}}+\frac{\sqrt{v}}{v^{3/2}}\right)\,dv \\
      \ln |x| &=- \int \left(v^{-3/2}+ \frac{1}{v}\right)\,dv \\
      \ln |x| &= \frac{2}{\sqrt{v}} - \ln |v| + C
    \end{align*}
    \(y=vx\), which means \(v=\frac{y}{x}\)

\vspace{0.5cm}

    Hence the implicit solution is
    \[-\frac{2\sqrt{x}}{\sqrt{y}}+\ln |y|=C\]
    or
    \[
    \ln |y|=\frac{2\sqrt{x}}{\sqrt{y}}+C.
    \]

    \end{minipage}
    }}
    \end{center}

\end{document}
