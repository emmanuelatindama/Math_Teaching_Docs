%========================================
% Quiz Preamble
%========================================

%----------- Encoding and Fonts ----------
\usepackage[T1]{fontenc}
\usepackage[utf8]{inputenc}
\usepackage{bm}
\usepackage{mathrsfs}
\usepackage{siunitx}
\usepackage{amsmath}  
\usepackage{mathtools}

%----------- Document Layout -------------
\usepackage{geometry}
\geometry{bottom=0.5in, top=0.75in, left=0.75in, right=2.75in}
\usepackage{ragged2e}
\usepackage{multicol}
\usepackage{lipsum}

%----------- Exam/Exercises --------------
\usepackage{exsheets}
\SetupExSheets{
  question/type=exam,
  solution/print=false,
  headings=block-subtitle,
  counter-format=qu
}
\NewQuSolPair{sagequestion}[name={\sage Question}]{sagesolution}
\newcommand{\solspace}[1]{\examspace*{#1}}
%\renewcommand{\solspace}[1]{{\tiny\textsf{\textcolor{red}{Blank: #1}}}}

%----------- Graphics and Figures --------
\usepackage{graphicx}
  \setkeys{Gin}{width=\linewidth,totalheight=\textheight,keepaspectratio}
  \graphicspath{{./img/}} 
\usepackage{subcaption}
\captionsetup{compatibility=false}
\usepackage{pgfplots}
\usepackage{pgfplotstable}
\usepgfplotslibrary{patchplots}
\pgfplotsset{compat=newest}
\pgfplotstableset{row sep=crcr}

% Custom PGFPlots styles
\makeatletter
\pgfplotsset{
  tufte axes/.style = {
    after end axis/.code = {
      \draw ({rel axis cs:0,0} -| {axis cs:\pgfplots@data@xmin,0})
        -- ({rel axis cs:0,0} -| {axis cs:\pgfplots@data@xmax,0});
      \draw ({rel axis cs:0,0} |- {axis cs:0,\pgfplots@data@ymin})
        -- ({rel axis cs:0,0} -| {axis cs:0,\pgfplots@data@ymax});
    },
    axis line style = {draw = none},
    tick align      = outside,
    tick pos        = left
  },
  textbook axes/.style = {
    enlargelimits = true,
    axis line style = {->},
    axis lines = middle,
    axis x line=middle,
    tick align      = outside,
    tick pos        = left,
    y label style={at={(ticklabel* cs:1.05)}, anchor=south},
    x label style={at={(ticklabel* cs:1.05)}, anchor=west}
  },
  every axis/.append style = {
    mark size = 0pt,
    width=0.9\linewidth
  }
}
\makeatother

%----------- Tables ----------------------
\usepackage{booktabs}
\usepackage{units}

%----------- Code / Verbatim -------------
\usepackage{fancyvrb}
\fvset{fontsize=\normalsize}

%----------- Utilities -------------------
\usepackage{xspace}
\usepackage{etoolbox}
\usepackage{lastpage}
\usepackage{pdfpages}

%----------- Hyperlinks ------------------
\usepackage{url}
\usepackage{hyperref}
\hypersetup{
  bookmarksnumbered=true,
  colorlinks=true
}

%========================================
% Custom Commands and Environments
%========================================

% Month + Year
\newcommand{\monthyear}{%
  \ifcase\month\or January\or February\or March\or April\or May\or June\or
  July\or August\or September\or October\or November\or
  December\fi\space\number\year
}

% Column enumerate
\newenvironment{colenumerate}[1][2]
  {\begin{multicols}{#1}\begin{compactenum}[(a)]}
  {\end{compactenum}\end{multicols}}

% Empty grid
\newcommand{\emptygrid}[3][1.0]{%
  \def\width{#2}
  \def\hauteur{#3}
  \begin{tikzpicture}[x=10mm,y=10mm,semitransparent,scale=#1,every node/.style={scale=#1}]
    \draw[step=1mm, line width=0.1mm, black!30!white] (0,0) grid (\width,\hauteur);
    \draw[step=5mm, line width=0.2mm, black!40!white] (0,0) grid (\width,\hauteur);
    \draw[step=50mm, line width=0.5mm, black!50!white] (0,0) grid (\width,\hauteur);
    \draw[step=10mm, line width=0.3mm, black!90!white] (0,0) grid (\width,\hauteur);
  \end{tikzpicture}
}

% Section formatting (no chapter number)
\renewcommand*\thesection{\arabic{section}}
\setcounter{secnumdepth}{1}
\titleformat{\section}[hang]{\normalfont\large\bfseries}{\thesection}{0.5em}{}
\titleformat{\subsection}[hang]{\normalfont\bfseries}{\thesubsection}{0.5em}{}

% Math operators
\newcommand{\vv}[1]{\ensuremath{\underline{#1}}}
\newcommand{\mm}[1]{\ensuremath{\bm{#1}}}
\newcommand{\red}[1]{\textcolor{red}{#1}}
\DeclareMathOperator{\tr}{tr}
\DeclareMathOperator{\rank}{rank}
\DeclareMathOperator{\nullity}{nullity}
\let\Re\relax
\DeclareMathOperator{\Re}{\mathbb{R}e}
\let\Im\relax
\DeclareMathOperator{\Im}{\mathbb{I}m}
\DeclareMathOperator{\laplace}{\mathscr{L}}
\DeclareMathOperator{\hstep}{\mathscr{U}}

% Math delimiters
\DeclarePairedDelimiter{\norm}{\lVert}{\rVert}
\DeclarePairedDelimiter{\abs}{\lvert}{\rvert}
\DeclarePairedDelimiter{\iprod}{\langle}{\rangle}
\DeclarePairedDelimiter{\braced}\{\}

% Small bmatrix
\newenvironment{smallbmatrix}{\left[ \begin{smallmatrix*}[r]}{\end{smallmatrix*} \right]}

% Misc environments
\newenvironment{weekintro}{\begin{quote}}{\end{quote}}

% Handy macros
\newcommand{\sage}{\href{http://www.sagemath.org}{\textsf{\textsc{SageMath}}}\xspace}
\newcommand{\ode}{ODE\xspace}

% Roman numeral
\makeatletter
\newcommand*{\rom}[1]{\expandafter\@slowromancap\romannumeral #1@}
\makeatother

% Number equations within section
\numberwithin{equation}{section}
