\chapter{Modeling with First Order \ode{}s}
A \textbf{model} is a mathematical description of a real-world situation.
\begin{itemize}
    \item Many processes in nature, engineering, economics, and biology change at a rate that depends on the current state of the system.
    \item These are often modeled by first-order differential equations:
    \[
    \frac{dy}{dt} = f(y,t)
    \]
    where
    \begin{itemize}
        \item $t$: independent variable (usually time),
        \item $y(t)$: dependent variable (population, temperature, amount, etc.),
        \item $f(y,t)$: rule describing the rate of change.
    \end{itemize}
\end{itemize}


\subsection*{Steps in Modeling}
  \begin{enumerate}
      \item Identify the system (population, chemical reaction, motion, etc.).
      \item Describe the rate of change in words (e.g., ``rate proportional to the population'').
      \item Translate into a differential equation.
      \item Solve the equation (analytically or numerically).
      \item Interpret the solution in context (units, growth/decay behavior).
  \end{enumerate}


\section{linear Models}
\subsection*{(a) Exponential Growth and Decay}
  Rate of change proportional to the current amount:
  \[
  \frac{dy}{dt} = ky, \quad k \in \mathbb{R}, \, y(t_0) = y_0.
  \]
  Solution:
  \[
  y(t) = y_0 e^{kt}.
  \]
  \textbf{Applications:} radioactive decay $(k<0)$, population growth $(k>0)$, carbon dating.

  \begin{example}
    \textbf{Exponential Growth (Population)}
    A bacteria culture has initial population $P(0)=500$. It doubles every 3 hours.

    \textbf{Solution:}
    \[
    \frac{dP}{dt} = kP, \quad P(t)=P_0e^{kt}.
    \]
    At $t=3$, $P(3)=1000=500e^{3k}$, so $e^{3k}=2 \Rightarrow k=\tfrac{\ln 2}{3}$.
    \[
    P(t) = 500 e^{(\ln 2/3)t}.
    \]
  \end{example}


\subsection*{(b) Newton's Law of Cooling/Heating}
  The rate of cooling is proportional to the temperature difference between an object and its surroundings:

  \noindent Rate proportional to difference from ambient temperature $T_a$:
  \[
  \frac{dT}{dt} = -k(T - T_a).
  \]
  Solution:
  \[
  T(t) = T_a + (T_0 - T_a)e^{-kt}.
  \]

  \begin{question}
    \textbf{Newton's Law of Cooling:}
    Coffee at $90^\circ C$ is placed in a $20^\circ C$ room. After 10 minutes it cools to $60^\circ C$. Find $T$ after 20 minutes.

    \textbf{Solution:}
    \[
    T(t) = 20 + 70e^{-kt}.
    \]
    At $t=10$, $60=20+70e^{-10k} \Rightarrow e^{-10k} = \tfrac{4}{7}$.
    Thus $k = \tfrac{1}{10}\ln \tfrac{7}{4}$.
    \[
    T(20) = 20 + 70\left(\frac{4}{7}\right)^2 \approx 42.9^\circ C.
    \]
  \end{question}

  \begin{question}
    A blacksmith removes a metal from the forge at \(750^{\circ}\)C.
    5 minutes later, the temperature is \(683^{\circ}\)C.
    How long would it take the metal to cool down to \(500^{\circ}\)C?
  \end{question}


\subsection*{(c) Mixing Problems}
  A mixing tank contains a volume of liquid with some solute (e.g., salt). If liquid flows in and out at certain rates, the concentration changes over time.

  \noindent Let $Q(t)$ be the amount of solute (kg) at time $t$:
  \[
  \frac{dQ}{dt} = \text{rate in} - \text{rate out}.
  \]
  If inflow concentration is $c_{in}$ and flow rate is $r$, and the tank volume is $V$, then:
  \[
  \frac{dQ}{dt} = r c_{in} - \frac{r}{V}Q.
  \]
  
  

  \begin{example}
    \textbf{Mixing Problem:}
    Tank: 100 L water with 20 g salt. Brine with 2 g/L enters at 5 L/min, mixture flows out at 5 L/min.

    \textbf{Solution:}
    Let $A(t)$ = salt amount.
    \[
    \frac{dA}{dt} = 10 - \frac{A}{20}.
    \]
    Solution:
    \[
    A(t) = 200 - 180e^{-t/20}.
    \]
  \end{example}


\subsection*{(d) Motion with Resistance}
  Falling object with air resistance proportional to velocity:
  \[
  m\frac{dv}{dt} = mg - kv,
  \]
  \[
  \frac{dv}{dt} = g - \frac{k}{m}v.
  \]
  Terminal velocity occurs when $\tfrac{dv}{dt} \to 0$.


\subsection*{(e) Series Circuits}
  Series circuits are a common source of first-order linear differential equations in physics and engineering.

  \textbf{I - Components of a Series Circuit}
  A typical circuit consists of:
  \begin{itemize}
      \item \(i(t)\): current ($A$)
      \item A resistor with resistance $R$ (measured in ohms $\Omega$).
      \item An inductor with inductance $L$ (measured in henries $H$).
      \item A capacitor with capacitance $C$ (measured in farads $F$).
      \item A voltage source $E(t)$ (measured in volts).
  \end{itemize}

  \textbf{II - Governing Equation (Kirchhoff’s Voltage Law)}
  Kirchhoff’s Voltage Law states that the sum of voltage drops around a closed loop equals the applied voltage (The sum of potential differences around a closed loop is zero):
  \[
  L \frac{di}{dt} + Ri + \frac{1}{C} \int i(t)\,dt = E(t).
  \]

  \noindent Taking derivatives, this can also be expressed as:
  \[
  L \frac{d^2 i}{dt^2} + R \frac{di}{dt} + \frac{1}{C} i(t) = \frac{dE}{dt}.
  \]

  \textbf{III Special Cases}
  \begin{itemize}
      \item \textbf{RL Circuit (no capacitor):}
      \[
      L \frac{di}{dt} + Ri = E(t).
      \]
      This is a first-order linear ODE in $i(t)$.
      \item \textbf{RC Circuit (no inductor):}
      \[
      R i + \frac{1}{C} \int i(t)\,dt = E(t),
      \]
      or equivalently, in terms of charge $q(t)$ on the capacitor,
      \[
      R \frac{dq}{dt} + \frac{1}{C}q = E(t).
      \]
  \end{itemize}

  \begin{example}
    \textbf{RL Circuit:}
    Suppose $L=1 \, H$, $R=2 \, \Omega$, and $E(t)=10 \, V$. Then:
    \[
    \frac{di}{dt} + 2i = 10.
    \]
    Solution:
    \[
    i(t) = 5 + Ce^{-2t}.
    \]
  \end{example}


\subsection*{Practice Exercises}
\begin{enumerate}
    \item Solve the RL circuit:
    \[
    L\frac{di}{dt} + Ri = E_0,
    \]
    with $L=2$, $R=4$, $E_0=20$, and $i(0)=0$.

    \item A tank initially contains $100$ L of pure water. Brine containing $0.5$ kg/L of salt enters at $5$ L/min, and the mixture leaves at the same rate. Find the amount of salt after $t$ minutes.

    \item A cup of coffee at $90^\circ C$ is left in a room at $20^\circ C$. After $10$ minutes the coffee has cooled to $70^\circ C$. Find the temperature after $20$ minutes.
\end{enumerate}









\section{Nonlinear Models}
  Nonlinear first-order differential equations arise in many real-world applications such as biology, physics, and chemistry. Unlike linear models, nonlinear models often exhibit more complex behavior such as saturation, thresholds, or blow-up in finite time. 

  General form:
  \[
  \frac{dy}{dt} = f(t, y), \quad \text{where $f$ is nonlinear in $y$.}
  \]

\subsection*{(a) Population Models (Logistic Growth)}
  A refinement of the exponential growth model that accounts for limited resources:
  \[
  \frac{dP}{dt} = rP\left(1 - \frac{P}{K}\right),
  \]
  where 
  \begin{itemize}
      \item $P(t)$ is the population at time $t$,
      \item $r > 0$ is the intrinsic growth rate,
      \item $K > 0$ is the carrying capacity.
  \end{itemize}

  \begin{example}
    Suppose a population grows logistically with $r=0.5$, $K=1000$, and initial population $P(0)=100$.  
    \[
    P(t) = \frac{1000}{1 + 9e^{-0.5t}}.
    \]
  \end{example}


  \subsection*{(b) Nonlinear Mixing Model}
  If concentration depends nonlinearly (e.g., saturation or reaction), the equation can become nonlinear.  

  \begin{example}
     \[
    \frac{dQ}{dt} = r c_{in} - \frac{r}{V}Q - kQ^2,
    \]
    where $kQ^2$ models a nonlinear reaction term.
  \end{example}


  \subsection*{(c) Newton’s Law of Cooling with Radiation}
    Newton’s law (linear) is:
    \[
    \frac{dT}{dt} = -k(T - T_s).
    \]
    If radiation is included (Stefan–Boltzmann law), we obtain:
    \[
    \frac{dT}{dt} = -k(T - T_s) - \sigma (T^4 - T_s^4),
    \]
    which is nonlinear.


    \begin{example}
      An object at $500\,K$ cools in surroundings at $300\,K$ with:
      \[
      \frac{dT}{dt} = -0.1(T-300) - 10^{-8}(T^4 - 300^4).
      \]
      This nonlinear ODE requires numerical methods (Euler, Runge–Kutta). The qualitative behavior: $T(t)\to 300$ as $t\to\infty$.
    \end{example}
    

\subsection*{5. Practice Exercises}
\begin{enumerate}
    \item \textbf{Exponential Growth:} A population of 100 triples to 300 after 5 years. Find population after 10 years.
    \item \textbf{Logistic Growth:} Fish population with carrying capacity 5000, initial 100, growth rate $k=0.2$. Write explicit solution.
    \item \textbf{Newton's Cooling:} A body at $37^\circ C$ in a $20^\circ C$ room cools to $30^\circ C$ after 1 hour. Find temperature after 2 hours.
    \item \textbf{Mixing:} Tank with 200 L pure water. Brine with 1 g/L enters at 4 L/min; mixture leaves at 2 L/min. Find salt content after 1 hour.
    \item \textbf{Motion with Resistance:} A 2 kg mass falls under gravity ($g=9.8$) with air resistance $k=4$. Find terminal velocity as $t\to\infty$.
\end{enumerate}


\section{Modeling with Systems of First Order \ode{}s}
  \subsection*{Predator-Prey Model (Lotka-Volterra)}
  A system of nonlinear equations describing interactions between two species:
  \[
  \begin{aligned}
  \frac{dx}{dt} &= \alpha x - \beta xy, \\
  \frac{dy}{dt} &= \delta xy - \gamma y,
  \end{aligned}
  \]
  where 
  \begin{itemize}
      \item $x(t)$ = prey population,
      \item $y(t)$ = predator population,
      \item $\alpha$ = prey growth rate,
      \item $\beta$ = predation rate,
      \item $\gamma$ = predator death rate,
      \item $\delta$ = predator reproduction rate from consuming prey.
  \end{itemize}


