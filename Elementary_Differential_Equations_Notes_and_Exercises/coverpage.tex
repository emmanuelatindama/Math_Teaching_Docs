%==========================
% FULL-PAGE COVER / COPYRIGHT
%==========================
\begin{titlepage}
\begin{fullwidth}
\thispagestyle{empty}
\setlength{\parindent}{0pt}
\setlength{\parskip}{1em}

\begin{center}
\vspace*{3cm}

% Title
{\Huge\bfseries MA2860 Elementary Differential Equations Notes and Problems}

\vspace{1.5cm}

% Author
{\Large Emmanuel Atindama, PhD Mathematics}

\vfill

% Copyright & License
Copyright \copyright\ \the\year\ \thanklessauthor

\par\smallcaps{Published by \thanklesspublisher}

\par\url{http://emmanuelatindama.github.io}

\par This work is licensed under a Creative Commons Attribution-NonCommercial-ShareAlike 4.0 International License:  
\url{http://creativecommons.org/licenses/by-nc-sa/4.0/}

\par\textit{\ver \quad Updated on \today}

\vfill

% Optional Preface summary
\begin{minipage}{0.85\textwidth}
\centering
\small
\textit{This text provides notes and problems that build on concepts in calculus for MA2860. It is open source and meant for students and instructors to use, adapt, and contribute. The computational component primarily consists of Sage code, and online symbolic calculators are provided for supplemental study.}
\end{minipage}

\vfill

% Location and date
University of Toledo, OH \\
\monthyear

\end{center}
\end{fullwidth}
\end{titlepage}

%==========================
% TABLE OF CONTENTS
%==========================
\cleardoublepage
\tableofcontents
\cleardoublepage

%==========================
% PREFACE
%==========================
\section*{Preface}
\addcontentsline{toc}{chapter}{Preface} % ensure it shows up in TOC
\begin{fullwidth}

\newthought{This text provides notes and problems} that build on concepts in calculus for \emph{MA2860 Elementary Differential Equations} at the University of Toledo. The book is a companion to various textbooks including:
\begin{itemize}
    \item Zill, Dennis G. \emph{A First Course in Differential Equations with Modeling Applications}, 10th ed. Cengage Learning, 2012.
    \item Boyce, William E., DiPrima Richard C. \emph{Elementary Differential Equations}, 8th ed. John Wiley \& Sons Inc., 2004.
\end{itemize}
Although it can be used with other texts as well.

\newthought{In making this starting resource of lesson planning \textbf{open source}}, I hope that each user (with code access) will contribute with notes, corrections, and exercises (with solutions). Users without code access may also email me any corrections. \emph{Professors may email to request code access.} Each week has a portion of the assignment reserved for preparatory homework and another portion for further discussion or exam preparation.

\newthought{Knowing that all people are made in the image of God with a potential for the highest good}, I hope that all users of this material will work to make this work of highest quality for all educators.

\noindent The idea for this work came from my professor in Dynamical Systems during my grad school years, Dr. Marko Budi\v{s}i\'{c} at Clarkson University. He encouraged students to compile class notes into one document for easy review. All credit to him.

\noindent This text uses a Differential Equations workbook compiled by Clarkson University faculty as a starting point, although the content is mostly different: K.~Black, D.~White, G.~Yao, J.~Martin, and M.~Budi\v{s}i\'{c}.

\newthought{The Computational component} of this book primarily consists of \sage code. No prerequisite experience is needed, as problems can be completed by following links and editing code in the browser. Essential manual for \sage is the textbook:
\begin{itemize}
    \item Bard, G.V. \emph{Sage for Undergraduates}. American Mathematical Society, 2015. Available at no cost in PDF form: \url{http://www.gregorybard.com/Sage.html}.
\end{itemize}

\newthought{Online tools} for studying involve symbolic calculators at links:
\textcolor{red}{CAUTION: do not rely heavily on them while studying!}
\begin{compactitem}
    \item \url{https://www.symbolab.com/solver/}
    \item \url{http://www.emathhelp.net/calculators/}
\end{compactitem}

\newthought{The goal} of this class is to teach the \textbf{principles} behind the calculations, so they can be built on and extended in future courses or practice, not convert people into human calculators. Therefore, emphasis is on algorithms, and memorization or \textbf{reliance on formulas and calculators is strongly discouraged}.

%==========================
% CONTRIBUTORS
%==========================
Contributing Authors:
\begin{colenumerate}
    \item Emmanuel Atindama, PhD Mathematics
    \item Second Author
    \item Third Author
    \item Fourth Author
\end{colenumerate}

\begin{flushright}
\newthought{Emmanuel Atindama} \\
\href{mailto:emmanuel.atindama@utoledo.edu}{emmanuelatindama.github.io} \\
Toledo, OH \\
\monthyear
\end{flushright}

\end{fullwidth}
