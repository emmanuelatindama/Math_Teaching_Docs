\PassOptionsToPackage{table}{xcolor} % load colortbl when xcolor is loaded - for coloring tables
\documentclass[10pt,twoside,sfsidenotes]{tufte-handout}

% --- Custom Preamble ---
%========================================
% Quiz Preamble
%========================================

%----------- Encoding and Fonts ----------
\usepackage[T1]{fontenc}
\usepackage[utf8]{inputenc}
\usepackage{bm}
\usepackage{mathrsfs}
\usepackage{siunitx}
\usepackage{amsmath}  
\usepackage{mathtools}

%----------- Document Layout -------------
\usepackage{geometry}
\geometry{bottom=0.5in, top=0.75in, left=0.75in, right=2.75in}
\usepackage{ragged2e}
\usepackage{multicol}
\usepackage{lipsum}

%----------- Exam/Exercises --------------
\usepackage{exsheets}
\SetupExSheets{
  question/type=exam,
  solution/print=false,
  headings=block-subtitle,
  counter-format=qu
}
\NewQuSolPair{sagequestion}[name={\sage Question}]{sagesolution}
\newcommand{\solspace}[1]{\examspace*{#1}}
%\renewcommand{\solspace}[1]{{\tiny\textsf{\textcolor{red}{Blank: #1}}}}

%----------- Graphics and Figures --------
\usepackage{graphicx}
  \setkeys{Gin}{width=\linewidth,totalheight=\textheight,keepaspectratio}
  \graphicspath{{./img/}} 
\usepackage{subcaption}
\captionsetup{compatibility=false}
\usepackage{pgfplots}
\usepackage{pgfplotstable}
\usepgfplotslibrary{patchplots}
\pgfplotsset{compat=newest}
\pgfplotstableset{row sep=crcr}

% Custom PGFPlots styles
\makeatletter
\pgfplotsset{
  tufte axes/.style = {
    after end axis/.code = {
      \draw ({rel axis cs:0,0} -| {axis cs:\pgfplots@data@xmin,0})
        -- ({rel axis cs:0,0} -| {axis cs:\pgfplots@data@xmax,0});
      \draw ({rel axis cs:0,0} |- {axis cs:0,\pgfplots@data@ymin})
        -- ({rel axis cs:0,0} -| {axis cs:0,\pgfplots@data@ymax});
    },
    axis line style = {draw = none},
    tick align      = outside,
    tick pos        = left
  },
  textbook axes/.style = {
    enlargelimits = true,
    axis line style = {->},
    axis lines = middle,
    axis x line=middle,
    tick align      = outside,
    tick pos        = left,
    y label style={at={(ticklabel* cs:1.05)}, anchor=south},
    x label style={at={(ticklabel* cs:1.05)}, anchor=west}
  },
  every axis/.append style = {
    mark size = 0pt,
    width=0.9\linewidth
  }
}
\makeatother

%----------- Tables ----------------------
\usepackage{booktabs}
\usepackage{units}

%----------- Code / Verbatim -------------
\usepackage{fancyvrb}
\fvset{fontsize=\normalsize}

%----------- Utilities -------------------
\usepackage{xspace}
\usepackage{etoolbox}
\usepackage{lastpage}
\usepackage{pdfpages}

%----------- Hyperlinks ------------------
\usepackage{url}
\usepackage{hyperref}
\hypersetup{
  bookmarksnumbered=true,
  colorlinks=true
}

%========================================
% Custom Commands and Environments
%========================================

% Month + Year
\newcommand{\monthyear}{%
  \ifcase\month\or January\or February\or March\or April\or May\or June\or
  July\or August\or September\or October\or November\or
  December\fi\space\number\year
}

% Column enumerate
\newenvironment{colenumerate}[1][2]
  {\begin{multicols}{#1}\begin{compactenum}[(a)]}
  {\end{compactenum}\end{multicols}}

% Empty grid
\newcommand{\emptygrid}[3][1.0]{%
  \def\width{#2}
  \def\hauteur{#3}
  \begin{tikzpicture}[x=10mm,y=10mm,semitransparent,scale=#1,every node/.style={scale=#1}]
    \draw[step=1mm, line width=0.1mm, black!30!white] (0,0) grid (\width,\hauteur);
    \draw[step=5mm, line width=0.2mm, black!40!white] (0,0) grid (\width,\hauteur);
    \draw[step=50mm, line width=0.5mm, black!50!white] (0,0) grid (\width,\hauteur);
    \draw[step=10mm, line width=0.3mm, black!90!white] (0,0) grid (\width,\hauteur);
  \end{tikzpicture}
}

% Section formatting (no chapter number)
\renewcommand*\thesection{\arabic{section}}
\setcounter{secnumdepth}{1}
\titleformat{\section}[hang]{\normalfont\large\bfseries}{\thesection}{0.5em}{}
\titleformat{\subsection}[hang]{\normalfont\bfseries}{\thesubsection}{0.5em}{}

% Math operators
\newcommand{\vv}[1]{\ensuremath{\underline{#1}}}
\newcommand{\mm}[1]{\ensuremath{\bm{#1}}}
\newcommand{\red}[1]{\textcolor{red}{#1}}
\DeclareMathOperator{\tr}{tr}
\DeclareMathOperator{\rank}{rank}
\DeclareMathOperator{\nullity}{nullity}
\let\Re\relax
\DeclareMathOperator{\Re}{\mathbb{R}e}
\let\Im\relax
\DeclareMathOperator{\Im}{\mathbb{I}m}
\DeclareMathOperator{\laplace}{\mathscr{L}}
\DeclareMathOperator{\hstep}{\mathscr{U}}

% Math delimiters
\DeclarePairedDelimiter{\norm}{\lVert}{\rVert}
\DeclarePairedDelimiter{\abs}{\lvert}{\rvert}
\DeclarePairedDelimiter{\iprod}{\langle}{\rangle}
\DeclarePairedDelimiter{\braced}\{\}

% Small bmatrix
\newenvironment{smallbmatrix}{\left[ \begin{smallmatrix*}[r]}{\end{smallmatrix*} \right]}

% Misc environments
\newenvironment{weekintro}{\begin{quote}}{\end{quote}}

% Handy macros
\newcommand{\sage}{\href{http://www.sagemath.org}{\textsf{\textsc{SageMath}}}\xspace}
\newcommand{\ode}{ODE\xspace}

% Roman numeral
\makeatletter
\newcommand*{\rom}[1]{\expandafter\@slowromancap\romannumeral #1@}
\makeatother

% Number equations within section
\numberwithin{equation}{section}
 % loads packages, custom commands, etc.

% Metadata
\date{} % suppress date

% Fancy headers/footers
\fancyhead[L]{MATH 2860 -- Q01 --- \monthyear}
\fancyhead[R]{\textbf{Name (Last, First):} \blank[width=3in]{}}
% \fancyhead[R]{\textbf{Name} (Last, First): \blank[width=3in]{} \\
% \textbf{Circle your TA}: Trevor \quad Yuchen \quad Samantha }
\fancyfoot[R]{\thepage/\pageref{LastPage}}
%\header{MA330F17 --- Q09 --- Nov 29}{}{\textbf{Name}: \fillin[Solutions][2in]}

%--------------------------
\begin{document}

% Reduce space around equations
\setlength\abovedisplayskip{2pt}
\setlength\belowdisplayskip{2pt}
\setlength\abovedisplayshortskip{2pt}
\setlength\belowdisplayshortskip{2pt}

%--------------------------
\begin{fullwidth}
  \begin{question} % Q01
    % ...question content...
    Given is the ODE \quad \(\displaystyle \frac{dy}{dt} = y^{2}(3-y)\).

    Fill in the information about the ODE:

    \begin{colenumerate}[2]
      \item Order: \blank[width=0.5in]{}
      \item Independent variable: \blank[width=0.5in]{}
      \item Dependent variable: \blank[width=0.5in]{}
      \item Time-dependence (circle one): \\ dependent \quad---\quad independent
    \end{colenumerate}
  \end{question}
\end{fullwidth}


\begin{fullwidth}
  \begin{question} % Q02
    % ...question content...
    The following Initial Value Problem involves a \textbf{linear} ODE:
    \begin{enumerate}[(i)]
      \item \[t \frac{dx}{dt} + x(t) = e^{t}, \quad x(2) = 0.\]
      \item \[\frac{1}{t^{2}} \frac{dx}{dt} + x(t) = \exp\left(t^{3}\right), \quad x(1) = 0.\]
    \end{enumerate}
    For one of the above IVPs, do the following:
    \begin{compactenum}[(a)]
      \item Convert the equation into the \textbf{standard form} and \textbf{label} all its components.
      \item Use \textbf{variation of parameters} to calculate the general solution.
      \item Compute the particular solution that satisfies the initial condition.
    \end{compactenum}

    {\footnotesize You will be graded both on \textbf{clarity of the process} to solution, and on \textbf{technical execution}. 
    Therefore, tell us what you are calculating, label your (sub)solutions and any partial results, and 
    \textbf{if you cannot complete the problem because of a calculation error}, explain in words what you would do to complete it.}
    \solspace{1.0in}
  \end{question}
  \end{fullwidth}



  \begin{question}
    % ...question content...
    Solve the following separable ODE:
    \begin{enumerate}[(i)]
      \item \[\frac{dy}{dx} = \frac{x^{2}}{y+1}\]
      \item \[\frac{dy}{dx} = \frac{\ln(x)}{x^{2}}\]
      \item \[\frac{dy}{dx} = x e^{x^{2}}\]
    \end{enumerate}
  \end{question}



  \begin{question}
    % ...question content...
    Solve the following linear ODE:
    \begin{enumerate}[(i)]
        \item \(y' + \frac{2}{x}y = \frac{\cos(x)}{x^2}\)
        \item \(y' + y \tan(x) = \sin(2x)\)
        \item \(y' + y \csc(x) = \cos(x)\)
    \end{enumerate}
  \end{question}



  \begin{question} % Q03-1
    % ...question content...
    The following \ode{} is exact.
    Calculate the \textbf{implicit} formula of its solution curves.
    \begin{enumerate}[(i)]
      \item \[(x^{2} + y^{2})y'(x) + 2xy+1  = 0.\]
      \item \[ \left( \frac{1}{2}x^{2} + xy \right) y'(x) + 2x^{3} + xy + \frac{1}{2}y^{2} = 0.\]
    Show all steps in your calculation.
    \end{enumerate}
  \end{question}
 


  \begin{question} % Q03-2
    % ...question content...
    The following ODE is linear, homogeneous 2nd order ODE. Using the characteristic equation, calculate the formula for its general solution
    \begin{enumerate}[(i)]
      \item \[y''(x) - 2 y'(x) - 3 y(x) = 0.\]
      \item \[\frac{d^{2}y}{dx^{2}} - 3 \frac{dy}{dx} = 0.\]
      \item \[y''(x) + 3 y'(x) + 2 y(x) = 0.\]
      \item \[y''(x) - 4 y'(x) + 4 y(x) = 0.\] 
    \end{enumerate}
    Show all steps in your calculation.
  \end{question}



  \begin{question} % Q04
    % ...question content...
    Given a differential equation with a damping coefficient \(c\) in it
    \[
      \frac{dy^{2}}{dt^{2}} + c \frac{dy}{dt} + 25 y(t) = 0
    \]
    perform both tasks below. Justify your work and explain your answers.\marginnote{In all cases, use the real-valued form of solution functions (sines/cosines instead of complex exponentials).}

    \begin{enumerate}[(a)]
      \item For damping \(c = 0\) calculate the solution that satisfies \(y(0) = 1\), \(y'(0) = 10\). \marginnote{Solutions to \(a\lambda^{2} + b\lambda + c = 0\) are \\
        \(\displaystyle \lambda_{1,2} = \frac{-b \pm \sqrt{b^{2} -4ac} }{2a}\)}
      \item Calculate the general solution for damping \(c = 6\). \textbf{Explain:} do solutions here in general oscillate or not? Do they grow/decay/stay the same?
    \end{enumerate}
  \end{question}



\begin{fullwidth}
  \begin{question} % Q05-1
    % ...question content...
    Given is the differential equation:
    \[
      y''(x) + 16 y(x) = 2x - 5e^{-2x}.
    \]
    \begin{enumerate}[(a)]
      \item Calculate its \textbf{general} solution. \marginnote{  In all cases, use \textbf{real basis} (sines, cosines, etc.) where appropriate. }
      \item Calculate the particular solution that satisfies \(y(0) = 0\), \(y'(0) = 0\).
    \end{enumerate}
  \end{question}
\end{fullwidth}
\clearpage
\thispagestyle{plain}
\begin{fullwidth}
  \begin{question} % Q05-2
    % ...question content...
    For the ODE with the same left-hand side as in Question 1,
    \[
      y''(x) + 16 y(x) = f(x)
    \]
    \textbf{suggest an example} of the input function \(f(x)\)  that would result in \textbf{resonant behavior}. (You do not need to solve for \(y(x)\).)

  \vspace{2em}

    {\Large  Resonating input \( f(x) = \)}

  \vspace{2em}

  \textbf{Explain your answer} by describing, in short, what resonance is.

  \blank[width=3.99\linewidth,linespread=2]{}
\end{question}
\end{fullwidth}
\clearpage



\begin{marginfigure}

  Laplace integral:

  $\laplace\braced*{f(t)} = \int_0^{\infty} e^{-st}
  f(t)dt $

  Properties, given that

  $\laplace\braced*{f(t)} = F(s)$, $\laplace\braced*{g(t)} = G(s)$:

  \begin{itemize}
    \item Linearity: 
      \\
      $
      \begin{aligned}
        \laplace&\braced*{c_1f(t)+c_2g(t)} \\&= c_1F(s) + c_2G(s)
      \end{aligned}
      $
    \item Convolution: $\laplace\braced*{f*g} = F(s)G(s)$, where
      \[
        (f*g)(t) = \int_0^t f(\tau)g(t-\tau)d\tau
      \]
    \item \(s\)-shift:  $\laplace\braced*{e^{at}f(t)} = F(s-a)$
    \item \(t\)-shift: $\laplace\braced*{\mathcal{U}(t-a)f(t-a)} = e^{-sa}F(s)$
    \item Derivative in \(t\): If $\laplace\braced*{y(t)} = Y(s)$, then
      \begin{align*}
        \laplace\braced*{y'(t)} &= sY(s)-y(0), \\
        \laplace\braced*{y''(t)} &= s^2Y(s) - sy(0) - y'(0)
      \end{align*}
  \end{itemize}

  \begin{center}
    \rowcolors{1}{white}{gray!15}
    \renewcommand{\arraystretch}{2}
    \begin{tabular}{ |>{\centering\arraybackslash}m{0.5in}|>{\centering\arraybackslash}m{0.5in}|}
      $f(t)$ & $F(s) = \laplace\braced*{f(t)}$\\\hline
      $1$ & $\frac{1}{s}$ \\
      $t^n$ & $\frac{n!}{s^{n+1}}$ \\
      $e^{at}$ & $\frac{1}{s-a}$ \\
      $\sin(\omega t)$ & $\frac{\omega}{s^2+\omega^2}$ \\
      $\cos(\omega t)$ & $\frac{s}{s^2+\omega^2}$ \\
      $e^{at}\sin(\omega t)$ & $\frac{\omega}{(s-a)^2+\omega^2}$ \\
      $e^{at}\cos(\omega t)$ & $\frac{s-a}{(s-a)^2+\omega^2}$ \\
      $\cosh(at)$ & $\frac{s}{s^2-a^2}$ \\
      $\sinh(at)$ & $\frac{a}{s^2-a^2}$ \\
    \end{tabular}
  \end{center}

\end{marginfigure}

\begin{question} % Q06-1
  % ...question content...
  Solve the following ODE \textbf{using Laplace transform}:
  \(\displaystyle
    y' + 2y = \frac{1}{3}e^{-t}, y(0) = -2
  \)
\end{question}

\vfill

\begin{question} % Q06-2
  % ...question content...
  Compute the inverse Laplace transform:
  \(\displaystyle \laplace^{-1}\braced*{\frac{6s}{s^{2} + 2s + 10}}\)
\end{question}
\clearpage




\begin{marginfigure}

  Laplace integral:

  $\laplace\braced*{f(t)} = \int_0^{\infty} e^{-st}
  f(t)dt $

  Properties, given that

  $\laplace\braced*{f(t)} = F(s)$, $\laplace\braced*{g(t)} = G(s)$:

  \begin{itemize}
    \item Linearity: 
      \\
      $
      \begin{aligned}
        \laplace&\braced*{c_1f(t)+c_2g(t)} \\&= c_1F(s) + c_2G(s)
      \end{aligned}
      $
    \item Convolution: $\laplace\braced*{f*g} = F(s)G(s)$, where
      \[
        (f*g)(t) = \int_0^t f(\tau)g(t-\tau)d\tau
      \]
    \item \(s\)-shift:  $\laplace\braced*{e^{at}f(t)} = F(s-a)$
    \item \(t\)-shift: $\laplace\braced*{\mathcal{U}(t-a)f(t-a)} = e^{-sa}F(s)$
    \item Derivative in \(t\): If $\laplace\braced*{y(t)} = Y(s)$, then
      \begin{align*}
        \laplace\braced*{y'(t)} &= sY(s)-y(0), \\
        \laplace\braced*{y''(t)} &= s^2Y(s) - sy(0) - y'(0)
      \end{align*}
  \end{itemize}

  \begin{center}
    \rowcolors{1}{white}{gray!15}
    \renewcommand{\arraystretch}{2}
    \begin{tabular}{ |>{\centering\arraybackslash}m{0.5in}|>{\centering\arraybackslash}m{0.5in}|}
      $f(t)$ & $F(s) = \laplace\braced*{f(t)}$\\\hline
      $1$ & $\frac{1}{s}$ \\
      $t^n$ & $\frac{n!}{s^{n+1}}$ \\
      $e^{at}$ & $\frac{1}{s-a}$ \\
      $\sin(\omega t)$ & $\frac{\omega}{s^2+\omega^2}$ \\
      $\cos(\omega t)$ & $\frac{s}{s^2+\omega^2}$ \\
      $e^{at}\sin(\omega t)$ & $\frac{\omega}{(s-a)^2+\omega^2}$ \\
      $e^{at}\cos(\omega t)$ & $\frac{s-a}{(s-a)^2+\omega^2}$ \\
      $\cosh(at)$ & $\frac{s}{s^2-a^2}$ \\
      $\sinh(at)$ & $\frac{a}{s^2-a^2}$ \\
    \end{tabular}
  \end{center}

\end{marginfigure}



\begin{question} % Q07-1
  % ...question content...
  Sketch as accurately as possible the impulse \( \delta( t - 5 ) \) and give its Laplace transform.

  \vfill
\end{question}



\begin{question} % Q07-2
  % ...question content...
  Compute and sketch the function \(g(t)\) defined as:
  \(\displaystyle g(t) = \laplace^{-1}\braced*{\frac{2}{s - 1}e^{-2s}} \)

  \vfill
\end{question}



\begin{question} % Q07-3
  Compute the Laplace transform of the function given by the graph.

  \begin{tikzpicture}
    \begin{axis}[textbook axes, xmajorgrids, ymajorgrids,height=2in,width=2in,
      xtick={0,...,4}, xlabel=\(t\),
      ytick={-1,0,...,8}, ylabel=\(f(t)\), ymin=-1, ymax=8,
      ]
      \addplot+[very thick,black] table {
        x f(x)\\
        0 0\\
        1 0\\
        1 1\\
        2 3\\
        3 5\\
        4 7\\
        4.2 7.4\\
      };

    \end{axis}
  \end{tikzpicture}

  \vfill
\end{question}



\begin{fullwidth}
  \begin{question} % Q08-1
    % ...question content...
    Given the following system of ODEs:
    \[
      \begin{aligned}
      \dot x &=  -5x + 6y\\
      \dot y &=  -3x + 4y
    \end{aligned}
    \]

    \begin{enumerate}[(a)]
      \item Determine the matrix \(\mm{M}\) in the matrix ODE \(\dot{\vv{v}}(t) = \mm{M}\vv{v}(t)\) 
        that corresponds to the system above if
        \(\vv{v}(t) =
        \begin{smallbmatrix}
          x(t) \\ y(t)
        \end{smallbmatrix}.
        \)
      \vspace{2in}
      \item Is the following candidate a solution of the matrix ODE from (a)? Show your work.
        \[
        \vv{v}(t) =
        \begin{bmatrix}
          -2 \\ -1
        \end{bmatrix} e^{-2t} +
        \begin{bmatrix}
          2 \\ 2
        \end{bmatrix} e^{t}
        \]
      \vfill
    \end{enumerate}
  \end{question}
\end{fullwidth}
\newpage

\fancyhead[R]{}
\begin{fullwidth}
  \begin{question} % Q08-2
    % ...question content...
    Use row reduction (Gauss elimination) to solve the following system of linear equations.
    If needed, use \(C_{1}, C_{2},\dots\) as free parameters in the solution. Label row operations in your solution.
    \begin{align*}
      -y + z -1 &= 0 \\
      -x + 2y -2z + 1 &= 0  \\
      2x -3y + 2z -1 & = 0 \\
    \end{align*}
  \end{question}
\end{fullwidth}



\begin{question} % Q09-1
  % ...question content...
  \begin{fullwidth}
    \begin{minipage}{0.65\linewidth}
      Matrix ODE \[\dot{\vv{x}} = \mm{M}\vv{x}\] is given by the matrix \(\mm{M}\), whose \textbf{eigenvalues and eigenvectors are already known}.
      \begin{enumerate}[(i)]
        \item
          \begin{equation*}
            \begin{aligned}
              \mm{M} &= 
              \begin{bmatrix*}[r]
                1/2 & -7/4 \\ -7 & 1/2
              \end{bmatrix*} \\
              \lambda_{1} = 4,\ \vv{v} =
              \begin{bmatrix*}[r]
                1 \\ -2
              \end{bmatrix*} \quad&\quad
              \lambda_{2} = -3,\ \vv{w} =
              \begin{bmatrix*}[r]
                1 \\ 2
              \end{bmatrix*}
            \end{aligned}
          \end{equation*}
        \item
          \begin{equation*}
            \begin{aligned}
              \mm{M} &= 
              \begin{bmatrix*}[r]
                -4/3 & 2/3 \\ 1/3 & -5/3
              \end{bmatrix*} \\
              \lambda_{1} = -1,\ \vv{v} =
              \begin{bmatrix*}[r]
                2 \\ 1
              \end{bmatrix*} \quad&\quad
              \lambda_{2} = -2,\ \vv{w} =
              \begin{bmatrix*}[r]
                -1 \\ 1
              \end{bmatrix*}
            \end{aligned}
          \end{equation*}
      \end{enumerate}
    \end{minipage}    
    \begin{minipage}{0.25\linewidth}\centering
      \emptygrid[0.7]{6}{6}
    \end{minipage}
  \end{fullwidth}
    
  \begin{compactenum}[(a)]
    \item  Sketch the phase portrait in the empty grid. Make sure to draw main directions of the flow and sketch in several solution curves.
    \item Is the fixed point stable or not? \textbf{Explain why.}

      \vspace{1in}

    \item Write out the \textbf{general solution} of the matrix ODE.

      \vspace{1in}

  \end{compactenum}
\end{question}

\newpage\fancyhead[R]{}
\begin{fullwidth} 
  \begin{question}
    Given \(\dot{\vv{x}} = \mm{A}\vv{x}\) with
    \[\mm{A} =
      \begin{bmatrix*}[r]
        \vary{-7 & -13 \\ 2 & 3}{-2 & -2 \\ 4 & 2}
      \end{bmatrix*}
    \]
    \begin{enumerate}[(a)]
      \item Compute eigenvalues and eigenvectors of \(\mm{A}\).

      \vfill

      \item Based on your calculations, would the solutions of  \(\dot{\vv{x}} = \mm{A}\vv{x}\)

      \begin{itemize}
        \item Oscillate --- Not Oscillate? \quad {\small (circle correct)} \quad
        \\ \textbf{Explain}:

        \vspace{1in}

        \item Grow --- Decay --- Neither \quad {\small (circle correct)} \quad \\ \textbf{Explain}:

        \vspace{1in}
      \end{itemize}
    \end{enumerate}
  \end{question}
\end{fullwidth}



\end{document}
