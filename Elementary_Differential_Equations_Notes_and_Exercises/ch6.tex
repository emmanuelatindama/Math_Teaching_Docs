\newpage
\chapter{Series Solutions to Second Order Linear \ode{}s}
\begin{fullwidth}
\begin{highlight}

Students were given the following problems and asked to work independently. Notes reflect a section of that work.

\begin{example}
  Let \(v \in \mathbb{C}^{N}\) be a complex vector, and \(v^{\ast}\) its complex-conjugate. Similarly, let \(\lambda, \lambda^{\ast} \in \mathbb{C}\) be two complex-conjugate scalars, similarly for \(\alpha, \alpha^{\ast}\)

  Find the formulas for the real and for the imaginary parts
  %\footnote{\(\Re z  = (z + z^{\ast})/2\)\\ \(\Im z  = (z - z^{\ast})/(2i)\)}  
  of the following expression: \[ \alpha e^{\lambda t} v \]

\end{example}

The exponential of a linear operator \(T\) is defined by a formal series
%\footnote{``Formal'' means that we, for now, ignore the question of convergence.}
\[e^{T} = \sum_{k=0}^{\infty} \frac{T^{k}}{k!}\]

\begin{example}
  From definition, prove that \(e^{T} = I\) if \(T = 0\).
\end{example}

\begin{example}
  From definition, prove that \(\left(e^{T}\right)^{-1} = e^{-T}\) for invertible \(T\).
\end{example}

\begin{example}
  From definition, prove that \(e^{PTP^{-1}} = Pe^{T}P^{-1}\) for invertible \(P\).
\end{example}

\begin{example}
  Let \(T\) have an eigenpair \((\lambda, v)\). Show that \(v\) is then also an eigenvector of \(T\), and compute the associated eigenvalue.
\end{example}

\begin{example}
  From definition, prove that \(\left(e^{T}\right)^{-1} = e^{-T}\) for invertible \(T\).
\end{example}

\begin{example}
  Use limit-definition of the derivative, and the definition of \(e^{T}\) to compute
  \[
    \frac{d}{dt}e^{At} =
  \]
\end{example}

\begin{example}
  For matrix \(A\), use definition of the power series to show that the expression \(x(t) = e^{At}x_{0}\) is a solution of the matrix ODE
  \[
    \dot x(t) = A x(t), \quad x(0) = x_{0}
  \]
\end{example}

\begin{example}
  Prove that if \(\norm{T}\) is finite,
  %\footnote{ Definition: \(\norm{T} = \sup_{x\not 0} \norm{Tx}/\norm{x}\) }
  so is \(\norm{e^{T}}\).
\end{example}

\end{highlight}
\end{fullwidth}

Group work
\begin{example}

    Let $v \in \mathbb{C} ^N $ be a complex vector, and $v^*$ its complex conjugate. Similarly, let $\lambda,\lambda^* \in \mathbb{C}$ be two complex-conjugate scalars, similarly for $\alpha, \alpha_*$ 
    
    
    Find the formulas for the real and for the imaginary parts of the following expression:
    
    
    
\begin{center}
    $\alpha e^{\lambda t}v$
\end{center}

\begin{proof}

    Let $v \in \mathbb{C}^N$ and $v^* -$ conjugate of complex vector $v$.
    
    
    $\alpha , \alpha^*$ and $\lambda , \lambda^*$ be two complex conjugate scalars.
  
  
    $v=v_1 + i v_2$
    
    
    $\lambda = \lambda_1 + i \lambda_2$ , $\alpha = \alpha_1 + i \alpha_2$
    
    
    $Z=\alpha e^{\lambda t}v$
    
    \textbf{Need to Find:} $\Re(z)$ and $\Im{z}$
    
   
\begin{align*}
 %f(x) &= (x+a)(x+b) \\
 \alpha e^{\lambda t}v &= e^{(\lambda_1 + i \lambda_2) t}v(v_1 + i v_2) \\
                       &= \alpha  e^{\lambda_1 t}[e^{i \lambda_2 t}v] \\
                       &= e^{\lambda_1 t}(\alpha v)e^{i \lambda_2 t} \\
                       &= e^{\lambda_1 t}([\alpha_1 + i \alpha_2][v_1 + i v_2])e^{i \lambda_2 t} \\
                       &= e^{\lambda_1 t}([\alpha_1v_1 - \alpha_2v_2]+i[\alpha_1v_2 + \alpha_2v_1])e^{i \lambda_2 t} \\
                       &= e^{\lambda_1 t}([\alpha_1v_1 - \alpha_2v_2]+i[\alpha_1v_2 + \alpha_2v_1])(\cos({\lambda_2 t)}+ i \sin{(\lambda_2 t)}) \\
     % &= x^2 + (a+b)x + ab
\end{align*} 
    
   Hence
   
   
   $\Re(z)= e^{\lambda_1 t}[(\alpha_1v_1 - \alpha_2v_2)\cos({\lambda_2 t}) - (\alpha_1v_2 + \alpha_2v_1])\sin{(\lambda_2 t)}]$
   
   
   $\Im{z}= e^{\lambda_1 t}[(\alpha_1v_1 - \alpha_2v_2) \sin{(\lambda_2 t)} - (\alpha_1v_2 + \alpha_2v_1) \cos({\lambda_2 t})] $
   
   
    
\end{proof}

\end{example}
\begin{definition}{Operator Exponentials}


The exponential of a linear operator T is defined by a formal series\footnote{"Formal" means that convergence was ignored.}

    \begin{equation}
        e^T=\sum^{inf}_{k=0} \frac{T^k}{k!}
    \end{equation}


\end{definition}
\begin{example}
  From definition, prove that $e^T=I$ if $T=0$
  
  \begin{proof}
  
  
    \textbf{NTS:}$ T=0 \xrightarrow{} e^T=I$
    
    
    Let $T=0$
\begin{align*}
 e^T &= \sum^{\inf}_{k=0} \frac{T^k}{k! }\\
      &= I + \frac{T}{1} + \frac{T^2}{2!} + \frac{T^3}{3!} + \cdots\\
      &= I + 0 + 0 + 0 + \cdots  \mid  \because T=0\\
      &= I
\end{align*}    
  
  \end{proof}
\end{example}



\begin{example}
  From definition, prove that $(e^T)^-1 = e^{-1}$ for invertible $T$
  
  
  \begin{proof}
    \textbf{NTS: }$(e^T)^-1 = e^{-T}$
    
    
    Let consider the expression.
\begin{align*}
 %f(x) &= (x+a)(x+b) \\
                         & (e^T)(e^{-T}) \\
                         &= \sum^{\inf}_{k=0} \frac{T^k}{k! } \times \sum^{\inf}_{k=0} \frac{(-T)^k}{k! } \\
                         &= [I + \frac{T}{1} + \frac{T^2}{2!} + \frac{T^3}{3!} + \cdots]\times [I - \frac{T}{1} + \frac{T^2}{2!} - \frac{T^3}{3!} + \cdots] \\
                         &= I + (T -T) + (-T^2 + \frac{T^2}{2} + \frac{T^2}{2}) + \cdots. \\
                         &= I + 0 + 0 + \cdots \\
                         &= I \\
     % &= x^2 + (a+b)x + ab
\end{align*}   
\begin{align*}
    i.e: (e^T)(e^{-T}) &= I \\
    (e^T)^-1 &= e^{-T}
\end{align*}   
    
  \end{proof}
  
\end{example}

\begin{example}
  From definition, prove that $e^{PTP^{-1}}=P e^T p^-1$ for invertible P.
  
  
  \begin{proof}
  \begin{align*}
  e^{PTP^-1} &= \sum_{k=0}^{inf} \frac{{(P T P^{-1})}^k}{k!} \\
             &= \sum_{k=0}^{inf} \frac{(P T P^{-1})(P T P^{-1})(P T P^{-1}) \cdots (P T P^{-1})}{k!} \\
             &= \sum_{k=0}^{inf} \frac{P T^k p^{-1}}{k!} \\
             &= p \sum_{k=0}^{inf} \frac{T^k}{k!} P^{-1} \\
     \end{align*}
  \end{proof}
\end{example}
