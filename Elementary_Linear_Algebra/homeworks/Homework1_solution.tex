\documentclass[a4paper,11pt,reqno]{amsart}

\usepackage[utf8]{inputenc}
\usepackage[foot]{amsaddr}
\usepackage{amsmath,amsfonts,amssymb,amsthm,mathrsfs,bm}
\usepackage[margin=0.95in]{geometry}
\usepackage{color}
\usepackage[dvipsnames]{xcolor}

\input{toc-config.tex}

\usepackage{mathtools,enumerate,mathrsfs,graphicx}
\usepackage{epstopdf}
\usepackage{hyperref}

\usepackage{latexsym}


\definecolor{CommentGreen}{rgb}{0.0,0.4,0.0}
\definecolor{Background}{rgb}{0.9,1.0,0.85}
\definecolor{lrow}{rgb}{0.914,0.918,0.922}
\definecolor{drow}{rgb}{0.725,0.745,0.769}

\usepackage{listings}
\usepackage{textcomp}
\lstloadlanguages{Matlab}%
\lstset{
    language=Matlab,
    upquote=true, frame=single,
    basicstyle=\small\ttfamily,
    backgroundcolor=\color{Background},
    keywordstyle=[1]\color{blue}\bfseries,
    keywordstyle=[2]\color{purple},
    keywordstyle=[3]\color{black}\bfseries,
    identifierstyle=,
    commentstyle=\usefont{T1}{pcr}{m}{sl}\color{CommentGreen}\small,
    stringstyle=\color{purple},
    showstringspaces=false, tabsize=5,
    morekeywords={properties,methods,classdef},
    morekeywords=[2]{handle},
    morecomment=[l][\color{blue}]{...},
    numbers=none, firstnumber=1,
    numberstyle=\tiny\color{blue},
    stepnumber=1, xleftmargin=10pt, xrightmargin=10pt
}

\numberwithin{equation}{section}
\synctex=1

\hypersetup{
    unicode=false, pdftoolbar=true, 
    pdfmenubar=true, pdffitwindow=false, pdfstartview={FitH}, 
    pdftitle={ELE2024 Coursework}, pdfauthor={A. Author},
    pdfsubject={ELE2024 coursework}, pdfcreator={A. Author},
    pdfproducer={ELE2024}, pdfnewwindow=true,
    colorlinks=true, linkcolor=red,
    citecolor=blue, filecolor=magenta, urlcolor=cyan
}


% CUSTOM COMMANDS
\renewcommand{\Re}{\mathbf{re}}
\renewcommand{\Im}{\mathbf{im}}
\newcommand{\R}{\mathbb{R}}
\newcommand{\N}{\mathbb{N}}
\newcommand{\C}{\mathbb{C}}
\newcommand{\lap}{\mathscr{L}}
\newcommand{\dd}{\mathrm{d}}
\newcommand{\smallmat}[1]{\left[ \begin{smallmatrix}#1 \end{smallmatrix} \right]}

%opening
\title[MATH 1890 (Elementary Linear Algebra)]{Homework 1 for MATH 1890 (Elementary Linear Algebra)}

\author[Emmanuel Atindama]{E. A. Atindama, PhD Mathematics}

\address[E. A. Atindama]{. Email addresses: \href{emmanuel.atindama@utoledo.edu}{emmanuel.atindama@utoledo.edu} 
% and 
% \href{mailto:a.student@qub.ac.uk}{a.student@qub.ac.uk}.
}
\thanks{Version 0.0.1. Last updated:~\today.}
\begin{document}

\maketitle

Due date: Sep 3rd at 10:00 EST in class

\subsection{Question Q1}\label{sec:q1}

Given the linear system
    \begin{align*}
        x_1 - 2x_2 + x_3 =& 0\\
           2x_2 - 8x_3  =& 8\\
         -4x_1 + 5x_2 + 9x_3 =& -9
    \end{align*}
\begin{enumerate}[a)]
    \item Rewrite the system in matrix notation.
    \item Using elementary row operations only, reduce the system to an upper triangular matrix.
    \item Using the upper triangular matrix, solve for \(x_1,\; x_2,\) and \(x_3\).
    
    \textbf{Note:} \emph{for an inconsistent system, you will not be able to solve the system}
    
    \item Is the system consistent?
\end{enumerate}

\begin{center}\setlength{\fboxsep}{10pt}\fcolorbox{yellow!20}{yellow!20}{\parbox{0.9\linewidth}{
\textbf{Solution}

\begin{enumerate}[a)]
\item Matrix form

$$
\begin{aligned}
\begin{bmatrix}
1 & -2 & 1\\
0 & 2 & -8\\
-4 & 5 & 9
\end{bmatrix}
\begin{bmatrix}x_1\\ x_2\\ x_3\end{bmatrix}
=
\begin{bmatrix}0\\ 8\\ -9\end{bmatrix}
\quad\Longleftrightarrow\quad
\left[\begin{array}{ccc|c}
1 & -2 & 1 & 0\\
0 & 2 & -8 & 8\\
-4 & 5 & 9 & -9
\end{array}\right].
\end{aligned}
$$

\item Row-reduce to an upper triangular matrix

$$
\begin{aligned}
\left[\begin{array}{ccc|c}
1 & -2 & 1 & 0\\
0 & 2 & -8 & 8\\
-4 & 5 & 9 & -9
\end{array}\right]
&\xrightarrow{R_3+4R_1 \rightarrow R_3}
\left[\begin{array}{ccc|c}
1 & -2 & 1 & 0\\
0 & 2 & -8 & 8\\
0 & -3 & 13 & -9
\end{array}\right] \\
&\xrightarrow{\; 2R_3+3R_2 \rightarrow2R_3\;}
\left[\begin{array}{ccc|c}
1 & -2 & 1 & 0\\
0 & 2 & -8 & 8\\
0 & 0 & 2 & 6
\end{array}\right]
\xrightarrow{\tfrac12 R_3 \rightarrow R_3}
\left[\begin{array}{ccc|c}
1 & -2 & 1 & 0\\
0 & 2 & -8 & 8\\
0 & 0 & 1 & 3
\end{array}\right].
\end{aligned}
$$

This is upper triangular.

\item Back-substitution

From the last row: $x_3=3$.

Second row: $2x_2-8x_3=8 \Rightarrow 2x_2-24=8 \Rightarrow x_2=16$.

First row: $x_1-2x_2+x_3=0 \Rightarrow x_1-32+3=0 \Rightarrow x_1=29$.

$$
\boxed{(x_1,x_2,x_3)=(29,\,16,\,3)}.
$$

\item Consistency

Yes—the system is \textbf{consistent} (no contradictory row such as $[0\ 0\ 0\mid c]$ with $c\neq 0$), and it has a \textbf{unique} solution.
\end{enumerate}
}}
\end{center}


\subsection{Question Q2}\label{sec:q2}
Given the linear system
    \begin{align*}
        x_2 - 4x_3 =& 8\\
           2x_1 - 3x_2 + 2x_3  =& 1\\
         5x_1 - 8x_2 + 7x_3 =& 1
    \end{align*}
\begin{enumerate}[a)]
    \item Rewrite the system in matrix notation.
    \item Using elementary row operations only, reduce the system to an upper triangular matrix.
    
    \textbf{Hint:} \emph{You may need to swap rows to achieve this.}
    
    \item Using the upper triangular matrix, solve for \(x_1,\; x_2,\) and \(x_3\).
    \textbf{Note:} \emph{for an inconsistent system, you will not be able to solve the system.}
    
    \item Is the system consistent?
\end{enumerate}


\begin{center}\setlength{\fboxsep}{10pt}\fcolorbox{yellow!20}{yellow!20}{\parbox{0.9\linewidth}{
\textbf{Solution}

\begin{enumerate}[a)]
\item \textbf{Matrix notation (augmented matrix).}

The given system can be written as $A\mathbf{x}=\mathbf{b}$ and the augmented matrix $[A\mid\mathbf{b}]$.

\[
A\mathbf{x}=\mathbf{b} \quad\Longleftrightarrow\quad
\begin{bmatrix}
0 & 1 & -4\\[4pt]
2 & -3 & 2\\[4pt]
5 & -8 & 7
\end{bmatrix}
\begin{bmatrix}x_1\\ x_2\\ x_3\end{bmatrix}
=
\begin{bmatrix}8\\1\\1\end{bmatrix},
\quad\text{so}\quad
\left[\begin{array}{ccc|c}
0 & 1 & -4 & 8\\[4pt]
2 & -3 & 2 & 1\\[4pt]
5 & -8 & 7 & 1
\end{array}\right].
\]

\item \textbf{Row operations to reach an upper triangular matrix.}

\noindent We want a nonzero pivot in the first column.So swap rows $R_1$ and $R_2$. 
Then eliminate the $5$ below in column 1: 
\(R_1\leftrightarrow R_2, \,
R_3 - \tfrac{5}{2}R_1 \rightarrow R_3
\)

\[
\left[\begin{array}{ccc|c}
0 & 1 & -4 & 8\\
2 & -3 & 2 & 1\\
5 & -8 & 7 & 1
\end{array}\right]
\longrightarrow
\left[\begin{array}{ccc|c}
2 & -3 & 2 & 1\\
0 & 1 & -4 & 8\\
5 & -8 & 7 & 1
\end{array}\right]
\longrightarrow
\left[\begin{array}{ccc|c}
2 & -3 & 2 & 1\\[4pt]
0 & 1 & -4 & 8\\[4pt]
0 & -\tfrac{1}{2} & 2 & -\tfrac{3}{2}
\end{array}\right].
\]

\[
R_3 + \tfrac{1}{2}R_2 \rightarrow R_3\; \text{to get the augmented system,}
\]

\[
\left[\begin{array}{ccc|c}
2 & -3 & 2 & 1\\[4pt]
0 & 1 & -4 & 8\\[4pt]
0 & 0 & 0 & \tfrac{5}{2}
\end{array}\right].
\]

This is upper triangular in the sense that all entries below the main diagonal are zero.

\item \textbf{Solve using the upper triangular matrix (back substitution).}

From the last row,:
\(
0\cdot x_1 + 0\cdot x_2 + 0\cdot x_3 = \tfrac{5}{2}, \Rightarrow 0 = \tfrac{5}{2}
\)
which is a contradiction. Thus,  the system is \emph{inconsistent} and has no solution.

\item \textbf{Is the system consistent?}

No! since we have \(
[0\; 0\; 0 \mid \tfrac{5}{2}]
\) in the last row.
\end{enumerate}

}}
\end{center}


\end{document}





