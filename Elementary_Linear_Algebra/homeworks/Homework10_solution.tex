\documentclass[a4paper,11pt,reqno]{amsart}

\usepackage[utf8]{inputenc}
\usepackage[foot]{amsaddr}
\usepackage{amsmath,amsfonts,amssymb,amsthm,mathrsfs,bm}
\usepackage[margin=0.95in]{geometry}
\usepackage{color}
\usepackage[dvipsnames]{xcolor}

\input{toc-config.tex}

\usepackage{mathtools,enumerate,mathrsfs,graphicx}
\usepackage{epstopdf}
\usepackage{hyperref}

\usepackage{latexsym}


\definecolor{CommentGreen}{rgb}{0.0,0.4,0.0}
\definecolor{Background}{rgb}{0.9,1.0,0.85}
\definecolor{lrow}{rgb}{0.914,0.918,0.922}
\definecolor{drow}{rgb}{0.725,0.745,0.769}

\usepackage{listings}
\usepackage{textcomp}
\lstloadlanguages{Matlab}%
\lstset{
    language=Matlab,
    upquote=true, frame=single,
    basicstyle=\small\ttfamily,
    backgroundcolor=\color{Background},
    keywordstyle=[1]\color{blue}\bfseries,
    keywordstyle=[2]\color{purple},
    keywordstyle=[3]\color{black}\bfseries,
    identifierstyle=,
    commentstyle=\usefont{T1}{pcr}{m}{sl}\color{CommentGreen}\small,
    stringstyle=\color{purple},
    showstringspaces=false, tabsize=5,
    morekeywords={properties,methods,classdef},
    morekeywords=[2]{handle},
    morecomment=[l][\color{blue}]{...},
    numbers=none, firstnumber=1,
    numberstyle=\tiny\color{blue},
    stepnumber=1, xleftmargin=10pt, xrightmargin=10pt
}

\numberwithin{equation}{section}
\synctex=1

\hypersetup{
    unicode=false, pdftoolbar=true, 
    pdfmenubar=true, pdffitwindow=false, pdfstartview={FitH}, 
    pdftitle={ELE2024 Coursework}, pdfauthor={A. Author},
    pdfsubject={ELE2024 coursework}, pdfcreator={A. Author},
    pdfproducer={ELE2024}, pdfnewwindow=true,
    colorlinks=true, linkcolor=red,
    citecolor=blue, filecolor=magenta, urlcolor=cyan
}


% CUSTOM COMMANDS
\renewcommand{\Re}{\mathbf{re}}
\renewcommand{\Im}{\mathbf{im}}
\newcommand{\R}{\mathbb{R}}
\newcommand{\N}{\mathbb{N}}
\newcommand{\C}{\mathbb{C}}
\newcommand{\lap}{\mathscr{L}}
\newcommand{\dd}{\mathrm{d}}
\newcommand{\smallmat}[1]{\left[ \begin{smallmatrix}#1 \end{smallmatrix} \right]}

%opening
\title[MATH 1890 (Elementary Linear Algebra)]{Homework 10 for MATH 1890 (Elementary Linear Algebra)}

\author[Emmanuel Atindama]{E. A. Atindama, PhD Mathematics}

\address[E. A. Atindama]{. Email addresses: \href{emmanuel.atindama@utoledo.edu}{emmanuel.atindama@utoledo.edu} 
% and 
% \href{mailto:a.student@qub.ac.uk}{a.student@qub.ac.uk}.
}
\thanks{Version 0.0.1. Last updated:~\today.}
\begin{document}

\maketitle
Due date: Wed Mar 19th 2025 / Th Mar 20th 2025 at 9am
\vspace{0.5cm}

\begin{enumerate}
    \item[\textbf{Q1:}] 
    Let \( {\mathcal{B}} = \left\{ \mathbf{b}_1, \mathbf{b}_2, \mathbf{b}_3 \right\} \) be the basis: 
    \(
    \mathbf{b}_1 = \begin{bmatrix} 1 \\ 0 \\ 1 \end{bmatrix}, \quad
    \mathbf{b}_2 = \begin{bmatrix} 0 \\ 1 \\ 1 \end{bmatrix}, \quad
    \mathbf{b}_3 = \begin{bmatrix} 1 \\ 1 \\ 0 \end{bmatrix}.
    \)
    Express the vector \( \mathbf{v} = \begin{bmatrix} 3 \\ 4 \\ 5 \end{bmatrix} \) in the \( {\mathcal{B}} \)-basis.
    
    \begin{center}\setlength{\fboxsep}{10pt}\fcolorbox{yellow!20}{yellow!20}{\parbox{0.9\linewidth}{
    \textbf{Solution}
    To find the \({\mathcal{B}}-\)coordinate \([\mathbf{v}]_{\mathcal{B}}\), we want to find the scalars \(c_1,c_2,c_3\) such that
    \[
    \mathbf{v} = c_1 \mathbf{b}_1 + c_2 \mathbf{b}_2 + c_3 \mathbf{b}_3.
    \]
    
    That is, 
    \(
    \begin{bmatrix} 3 \\ 4 \\ 5 \end{bmatrix}
    = c_1 \begin{bmatrix} 1 \\ 0 \\ 1 \end{bmatrix}
    + c_2 \begin{bmatrix} 0 \\ 1 \\ 1 \end{bmatrix}
    + c_3 \begin{bmatrix} 1 \\ 1 \\ 0 \end{bmatrix}
    \quad \Rightarrow \quad 
    \begin{bmatrix} 1 & 0 & 1 \\ 0 & 1 & 1 \\ 1 & 1 & 0 \end{bmatrix}
    \begin{bmatrix} c_1 \\ c_2 \\ c_3 \end{bmatrix}
    = \begin{bmatrix} 3 \\ 4 \\ 5 \end{bmatrix}
    \)
  
    Then we perform the row reduction,
    \begin{eqnarray*}
    \begin{bmatrix}
    1 & 0 & 1 & | & 3 \\
    0 & 1 & 1 & | & 4 \\
    1 & 1 & 0 & | & 5
    \end{bmatrix}
    \quad
    \text{Swap } R_1\text{ and }R_3&
    \quad
    \begin{bmatrix}
    1 & 1 & 0 & | & 5 \\
    0 & 1 & 1 & | & 4 \\
    1 & 0 & 1 & | & 3
    \end{bmatrix}\\
    -R_1 + R_3 \rightarrow R_3&
    \begin{bmatrix}
    1 & 1 & 0 & | & 5 \\
    0 & 1 & 1 & | & 4 \\
    0 & -1 & 1 & | & -2
    \end{bmatrix}\\
     R_2 + R_3 \rightarrow R_3&
    \begin{bmatrix}
    1 & 1 & 0 & | & 5 \\
    0 & 1 & 1 & | & 4 \\
    0 & 0 & 2 & | & 2
    \end{bmatrix}
    \end{eqnarray*}

    \(
     c_1 = 2, \; c_2 = 3, \;  c_3 = 1
    \).
    So, \([\mathbf{v}]_{\mathcal{B}} = \begin{bmatrix} 2 \\ 3 \\ 1 \end{bmatrix}\)
    }}
    \end{center}


    \item[\textbf{Q2:}]
    Let:
    \(
    {\mathcal{B}} = \left\{ \begin{bmatrix} 1 \\ 2 \\ 1 \end{bmatrix}, \begin{bmatrix} 0 \\ 1 \\ 2 \end{bmatrix}, \begin{bmatrix} 2 \\ 0 \\ 3 \end{bmatrix} \right\}, \quad
    {\mathcal{C}} = \left\{ \begin{bmatrix} 2 \\ 1 \\ 1 \end{bmatrix}, \begin{bmatrix} 1 \\ 3 \\ 4 \end{bmatrix}, \begin{bmatrix} 0 \\ 1 \\ 1 \end{bmatrix} \right\}.
    \)
    Find the change of basis matrix \( P_{{\mathcal{B}} \to {\mathcal{C}}} \).
    
    \begin{center}\setlength{\fboxsep}{10pt}\fcolorbox{yellow!20}{yellow!20}{\parbox{0.9\linewidth}{
    \textbf{Solution}
    To find the change of basis matrix \( P_{{\mathcal{B}} \to {\mathcal{C}}} \), we want to express each vector in the \( \mathcal{B} \)-basis in terms of the \( \mathcal{C} \)-basis. 

In other words, for each \( \mathbf{b}_i \in \mathcal{B} \), we find the coordinates \( [\mathbf{b}_i]_{\mathcal{C}} \) by solving:

\[
\mathbf{b}_i = c_{i1} \mathbf{c}_1 + c_{i2} \mathbf{c}_2 + c_{i3} \mathbf{c}_3 \qquad \textbf{for }i = 1, 2, 3.
\]
 

\begin{eqnarray*}
\text{For }  \mathbf{b}_1 = \begin{bmatrix} 1 \\ 2 \\ 1 \end{bmatrix}, \;\;
    \begin{bmatrix} 
    2 & 1 & 0 & | & 1 \\ 
    1 & 3 & 1 & | & 2 \\ 
    1 & 4 & 1 & | & 1 
    \end{bmatrix}
    \begin{array}{c}
      -\frac{1}{2} R_1 + R_2 \to R_2\\  
      - \frac{1}{2} R_1 + R_3 \to R_3 
    \end{array}& 
    \begin{bmatrix} 
    2 & 1 & 0 & | & 1 \\ 
    0 & \frac{5}{2} & 1 & | & \frac{3}{2} \\ 
    0 & \frac{7}{2} & 1 & | & \frac{1}{2} 
    \end{bmatrix}\\
    - \frac{7}{5} R_2 + R_3 \to R_3&
    \begin{bmatrix} 
    2 & 1 & 0 & | & 1 \\ 
    0 & \frac{5}{2} & 1 & | & \frac{3}{2} \\ 
    0 & 0 & \frac{-2}{5} & | & -\frac{8}{5} 
    \end{bmatrix}
    \end{eqnarray*}
    
    Thus, \( [\mathbf{b}_1]_{\mathcal{C}} = \begin{bmatrix} \;\;1 \\ -1 \\ \;\;4 \end{bmatrix} \).
    
     
    \begin{eqnarray*}
    \text{For } \mathbf{b}_2 = \begin{bmatrix} 0 \\ 1 \\ 2 \end{bmatrix}, \;\;
    \begin{bmatrix} 
    2 & 1 & 0 & | & 0 \\ 
    1 & 3 & 1 & | & 1 \\ 
    1 & 4 & 1 & | & 2 
    \end{bmatrix}
    \begin{array}{c}
      -\frac{1}{2} R_1 + R_2 \to R_2\\  
      - \frac{1}{2} R_1 + R_3 \to R_3 
    \end{array}& 
    \begin{bmatrix} 
    2 & 1 & 0 & | & 0\\ 
    0 & \frac{5}{2} & 1 & | & 1 \\ 
    0 & \frac{7}{2} & 1 & | & 2 
    \end{bmatrix}\\
    - \frac{7}{5} R_2 + R_3 \to R_3&
    \begin{bmatrix} 
    2 & 1 & 0 & | & 0 \\ 
    0 & \frac{5}{2} & 1 & | & 1 \\ 
    0 & 0 & \frac{-2}{5} & | & \frac{3}{5} 
    \end{bmatrix}
    \end{eqnarray*}

    Thus, \( [\mathbf{b}_2]_{\mathcal{C}} = \begin{bmatrix} -\frac{1}{2} \\ \;\; 1 \\ -\frac{3}{2} \end{bmatrix} \).


\begin{eqnarray*}
    \text{For } \mathbf{b}_3 = \begin{bmatrix} 2 \\ 0 \\ 3 \end{bmatrix}, \;\;
    \begin{bmatrix} 
    2 & 1 & 0 & | & 2 \\ 
    1 & 3 & 1 & | & 0 \\ 
    1 & 4 & 1 & | & 3 
    \end{bmatrix}
    \begin{array}{c}
      -\frac{1}{2} R_1 + R_2 \to R_2\\  
      - \frac{1}{2} R_1 + R_3 \to R_3 
    \end{array}& 
    \begin{bmatrix} 
    2 & 1 & 0 & | & 2 \\ 
    0 & \frac{5}{2} & 1 & | & -1 \\ 
    0 & \frac{7}{2} & 1 & | & 2 
    \end{bmatrix}\\
    - \frac{7}{5} R_2 + R_3 \to R_3&
    \begin{bmatrix} 
    2 & 1 & 0 & | & \;\;2 \\ 
    0 & \frac{5}{2} & 1 & | & -1 \\ 
    0 & 0 & \frac{-2}{5} & | & \;\;\frac{17}{5} 
    \end{bmatrix}
    \end{eqnarray*}

    Thus, \( [\mathbf{b}_3]_{\mathcal{C}} = \begin{bmatrix} -\frac{1}{2} \\ \;\;3 \\ -\frac{17}{2} \end{bmatrix} \).

    
    \[
    \text{ Therefore the change of basis matrix is }\quad
    P_{\mathcal{B} \to \mathcal{C}} = 
    \begin{bmatrix}
    \;\;1 & -\frac{1}{2} & -\frac{1}{2} \\
    -1 & \;\;1 & \;\;3 \\
    \;\;4 & -\frac{3}{2} & -\frac{17}{2}
    \end{bmatrix}
    \]
    }}

    \end{center}

    \item[\textbf{Q3:}]
    If the transformation matrix of \( T \) in the basis \( {\mathcal{H}} \) with basis vectors \( \mathbf{h}_1 = (3,1) \) and \( \mathbf{h}_2 = (0,1) \) is:
    \[
    A = \begin{bmatrix} 4 & 2 \\ 3 & 1 \end{bmatrix},
    \]
    what is its representation in a new basis \( {\mathcal{B}} \) with basis vectors \( \mathbf{b}_1 = (3,1) \) and \( \mathbf{b}_2 = (2,-1) \)?

    \begin{center}\setlength{\fboxsep}{10pt}\fcolorbox{yellow!20}{yellow!20}{\parbox{0.9\linewidth}{
    \textbf{Solution}
    
    \vspace{0.3cm}
    
    To find the transformation matrix in terms of the \(\mathcal{B}\)-coordinate system,
    \begin{itemize}
        \item First, find the transformation matrix \( P_{\mathcal{B} \to \mathcal{H}}\) that converts the vectors in the \(\mathcal{B}\) coordinate system to the \(\mathcal{H}\).
        \item Now that the vector is in \(\mathcal{H}\), apply the transformation \( T \) of the \(\mathcal{H}\)-coordinate system.
        \item Finally, apply the inverse transformation matrix \( P^{-1}_{\mathcal{B} \to \mathcal{H}}\) to return it to the \(\mathcal{B}\)-coordinate system.
    \end{itemize}
    
    \vspace{0.3cm}
    
    Express each vector in \( {\mathcal{B}} \) in terms of \( {\mathcal{H}} \):
    \begin{eqnarray*}
    \begin{bmatrix} 3 \\ 1 \end{bmatrix} = a_1 \begin{bmatrix} 3 \\ 1 \end{bmatrix} + a_2 \begin{bmatrix} 0 \\ 1 \end{bmatrix} \;\Rightarrow&\;
    \begin{bmatrix} 3 \\ 1 \end{bmatrix} = \begin{bmatrix} 3  & 0 \\ 1 & 1\end{bmatrix}
    \begin{bmatrix} a_1 \\ a_2 \end{bmatrix}, \;\; \text{ solve for } a_1 \text{ and } a_2\\
    &\\
    \begin{bmatrix} \;\;2 \\ -1 \end{bmatrix} = c_1 \begin{bmatrix} 3 \\ 1 \end{bmatrix} + c_2 \begin{bmatrix} 0 \\ 1 \end{bmatrix} \;\Rightarrow&\;
    \begin{bmatrix} \;\;2 \\ -1 \end{bmatrix} = \begin{bmatrix} 3  & 0 \\ 1 & 1\end{bmatrix}
    \begin{bmatrix} c_1 \\ c_2 \end{bmatrix}\;\;, \text{ solve for } c_1 \text{ and } c_2.
    \end{eqnarray*}

    \vspace{0.3cm}
    
    Solving both systems, we get 
    \([\mathbf{b}_1]_{\mathcal{H}} = \begin{bmatrix} a_1 \\ a_2 \end{bmatrix} =  \begin{bmatrix} 1\\0 \end{bmatrix}\)
    and 
    \([\mathbf{b}_2]_{\mathcal{H}} = \begin{bmatrix} c_1 \\ c_2 \end{bmatrix} = \begin{bmatrix} \;\;\frac{2}{3}\\-\frac{5}{3} \end{bmatrix}\).
    So,
    \(
    P_{{\mathcal{B}} \to {\mathcal{H}}} =
    \begin{bmatrix} 1 & \;\;\frac{2}{3} \\ 0 & -\frac{5}{3} \end{bmatrix}
    \) 
    and  \(
    P^{-1}_{{\mathcal{B}} \to {\mathcal{H}}} =
    \begin{bmatrix} 1 & \;\;\frac{2}{5} \\ 0 & -\frac{3}{5} \end{bmatrix}
    \)

    \vspace{0.3cm}
    
    The transformation matrix in the new basis \( {\mathcal{B}} \) is
    \[
    A_{\mathcal{B}} = 
    \begin{bmatrix} 1 & \;\;\frac{2}{5} \\ 0 & -\frac{3}{5} \end{bmatrix}
    \begin{bmatrix} 4 & 2 \\ 3 & 1 \end{bmatrix}
    \begin{bmatrix} 1 & \;\;\frac{2}{3} \\ 0 & -\frac{5}{3} \end{bmatrix}
    \]
    
    }}
    \end{center}
\end{enumerate}





\end{document}
